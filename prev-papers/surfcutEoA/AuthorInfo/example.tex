%%%%%%%%%%%%%%%%%%%%%%%%%% example.tex %%%%%%%%%%%%%%%%%%%%%%%%%%%
%
% sample input file for your contribution to Springer ENCYCLOPEDIAs
%
% Use this file as a template for your own input.
%
%%%%%%%%%%%%%%%%%%%%%%%% Springer-Verlag %%%%%%%%%%%%%%%%%%%%%%%%%%


% RECOMMENDED %%%%%%%%%%%%%%%%%%%%%%%%%%%%%%%%%%%%%%%%%%%%%%%%%%%%%
\documentclass[natbib]{svcyclop}

% \documentclass[natbib]{svcyclop}
% call for the natbib-system if you wish to have sophisticated citations,
% remember to adapt the references at the end of your contribution then
% see explanation at the end of this file.
%
% A detailed explanation and demonstration of the natbib system can
% be found at
%     http://merkel.zoneo.net/Latex/natbib.php
%
% If you use BibTeX to manage your citation use the style "spbasic.bst"
% that is "natbib" aware.

\usepackage{graphicx}    % standard LaTeX graphics tool
                         % when including figure files

\usepackage{amsfonts,amsmath} % typesetting complicated math

% Hint:
% Use the "hyperref" package in the preamble
% if you want to supply active links within
% your text; e.g. at the \CrossRef section
\usepackage{hyperref}

% Hint:
% Use the package "url" in the preamble
% to avoid problems with special characters
% used in your e-mail or web address in the
% \author or the \institute field below
\usepackage{url}
\urlstyle{tt}

% etc.

%%%%%%%%%%%%%%%%%%%%%%%%%%%%%%%%%%%%%%%%%%%%%%%%%%%%%%%%%%%%%%%%%%%

\begin{document}

% Please specify your Entry Title in one continuous line
\title{Circuit Placement}

% Please specify the Names of Entry Author(s)
\author{Andrew A. Kennings\inst{1}
\and
Igor I. Markov\inst{2}}
% if there is more than one author, use "\and" in between
% the particular author lines
%
% add "\inst{<No.>}" directly after the author name to indicate
% the mapping to the relevant affiliations/addresses in the
% forthcoming "\institute" field
%
% see also the example contribution file "example.tex" for detailed info

% Please specify your Address/Affiliation and use "\\" as EOL for structuring
% specify Phone / Fax / e-mail if applicable, use "\\" as EOL for structuring
% more than one address/affiliation? separate each two entries with
% a line containing "\and"
\institute{
Department of Electrical and Computer Engineering,
University of Waterloo, Waterloo, ON, Canada
\and
Department of Electrical Engineering and Computer
Science, University of Michigan, Ann Arbor, MI, USA
}
% make sure to give at least as many institute addresses
% as the mapping to the authors requires;
% the numbering is done automatically by the "\and" command
% see also the example contribution file "example.tex" for detailed info

% Please specify Years and Authors of Summarized Original Work
% use one continuous line for each paper; use semicolon to separate
% the year from the last names; don't use "and" between author names
% adhere to the following format:
% "Year of paper"; "last names of authors" separated by commas without "and"
% use "\\" as EOL for structuring
\sumoriwork{2000; Caldwell, Kahng, Markov\\
2002; Kennings, Markov\\
2006; Kennings, Vorwerk}
% see also the example contribution file "example.tex"
% for detailed information
%

% Please specify the Keywords - use semicolon for separation
\keywords{EDA; Netlist; Layout; Min-cut placement; Min-cost maxflow; Analytical placement; Mathematical programming}

\maketitle  % completes, checks, and typesets the contribution's head

% you have 7 predefined headings at hand you can use throughout your contribution;
% just call the relevant commands instead of generating a new "\section" for them;
% the next two headings are mandatory:
% \ProbDef   -   Problem Definition
% \KeyRes    -   Key Results
%
% the following heading commands may be used as and when required:
% \Applic    -   Applications
% \OpenProb  -   Open Problems
% \ExpRes    -   Experimental Results
% \DataSets  -   URLs to Code and Data Sets
% \CrossRef  -   Cross References


\ProbDef

This problem is concerned with efficiency determining constrained
positions of objects while minimizing a measure of interconnect between
the objects, as in a physical layout of integrated circuits, commonly
done in 2-dimensions. While most formulations are NP-hard, modern circuits
are so large that practical algorithms for placement
must have near-linear runtime and memory requirements,
but not necessarily produce optimal solutions. While early
software for circuit placement was based on Simulated Annealing,
research in algorithms identified more scalable
techniques which are now being adopted in the Electronic
Design Automation industry.

One models a circuit by a hypergraph $G_h(V_h,E_h)$ with (I) vertices
$V_h=\{v_1,\ldots,v_n\}$ representing logic gates,
standard cells, larger modules, or fixed I/O pads and (ii)
$E_h=\{e_1,\ldots,e_m\}$ representing connections
between modules. Every incident pair of a vertex and a hyperedge
connect through a pin for a total of $P$ pins in the
hypergraph. Each vertex $v_i\in V_h$ has width $w_i$, height
$h_i$ and area $A_i$. Hyperedges may also be weighted. Given
$G_h$, circuit placement seeks center positions $(x_i,y_i)$ for vertices
that optimize a hypergraph-based objective subject
to constraints (see below). A placement is captured by
$\mathrm{x}= (x_1,\ldots,x_n\}$ and $\mathrm{y}= (y_1,\ldots,y_n\}$.

\subsubsection{Objective} Let $C_k$ be the index set of the hypergraph
vertices incident to hyperedge $e_k$. The total halfperimeter
wirelength $(\mathrm{HPWL})$ of the circuit hypergraph
is given by $(\mathrm{HPWL}(G_h) = \sum_{e_k\in E_h}\mathrm{HPWL}(e_k) =
\sum_{e_k\in E_h}\left[\max_{i,j\in C_k}|x_i - x_j|+\max_{i,j\in C_k}
|y_i - y_j|\right]$.

$\mathrm{HPWL}$ is piece-wise linear, separable in the $x$ and $y$ directions,
convex, but not strictly convex. Among many
objectives for circuit placement, it is the simplest and
most common.

\subsection{Constraints}
\begin{enumerate}
\item \textbf{No overlap.} The area occupied by any two vertices cannot
overlap; i. e., either $|x_i-x_j|\ge \frac12(w_i + w_j)$ or
$|y_i-y_j|\ge \frac12(h_i + h_j), \forall v_i,v_j\in V_h$.
\item \textbf{Fixed outline.} Each vertex $v_i\in V_h$ must be placed entirely
within a specified rectangular region bounded by
$x_{\textrm{min}}(y_\textrm{min})$ and $x_{\textrm{max}}(y_\textrm{max})$
 which denote the left (bottom)
and right (top) boundaries of the specified region.
\item \textbf{Discrete slots.} There is only a finite number of discrete
positions, typically on a grid. However, in large-scale
circuit layout, slot constraints are often ignored during
\emph{global placement}, and enforced only during \emph{legalization}
and \emph{detail placement}.
\end{enumerate}

Other constraints may include alignment, minimum and
maximum spacing, etc. Many placement techniques temporarily
relax overlap constraints into density constraints
to avoid vertices clustered in small regions. A $m\times n$ regular
bin structure$ $B is superimposed over the fixed outline
and vertex area is assigned to bins based on the positions
of vertices. Let $D_{ij}$ denote the density of bin $B_{ij}\in B$, defined
as the total cell area assigned to bin $B_{ij}$ divided by
its capacity. Vertex overlap is limited implicitly by $D_{ij}\le
K;\forall B_{ij}\in B$, for some $K \le 1$ (density target).

\subsection{Problem 1 (Circuit Placement)}
{\itshape
\textsc{INPUT}: Circuit hypergraph $G_h(V_h,E_h)$ and a fixed outline
for the placement area.
\textsc{OUTPUT}: Positions for each vertex $v_i\in V_h$ such that
(1) wirelength is minimized and (2) the area-density constraints
$D_{ij}\le K$ are satisfied for all $B_{i j}\in B$.}

\KeyRes

An unconstrained optimal position of a single placeable
vertex connected to fixed vertices can be found in linear
time as the median of adjacent positions \cite{KenVor06}. Unconstrained
$\mathrm{HPWL}$ minimization for multiple placeable vertices
can be formulated as a linear program \cite{KenMar02,RedCho06}. For each
$e_k \in  E_h$, upper and lower bound variables $U_k$ and $L_k$ are
added. The cost of $e_k$ ($x$-direction only) is the difference
between $U_k$ and $L_k$. Each $U_k(L_k)$ comes with $p_k$ inequality
constraints that restricts its value to be larger (smaller)
than the position of every vertex $i\in C_k$. A hypergraph
with $n$ vertices and $m$ hyperedges is represented by a linear
program with $n + 2m$ variables and $2P$ constraints.

Linear programming has poor scalability, and integrating
constraint-tracking into optimization is difficult.
Other approaches include non-linear optimization and
partitioning-based methods.

\subsection{Combinatorial Techniques
for Wirelength Minimization}
The no-overlap constraints are not convex and cannot be
directly added to the linear program for $\mathrm{HPWL}$ minimization.
Such a program is first solved directly or by casting its
dual as an instance of the min-costmax-flow problem \cite{TanTiaWon05}.
Vertices often cluster in small regions of high density. One
can lower-bound the distance between closely-placed vertices
with a single linear constraint that depends on the relative
placement of these vertices \cite{RedCho06}. The resulting optimization
problem is incrementally re-solved, and the process
repeats until the desired density is achieved.

The \emph{min-cut placement} technique is based on balanced
min-cut partitioning of hypergraphs and is more focused
on density constraints \cite{RoyAdyPapMar06}. Vertices of the initial hypergraph
are first partitioned in two similar-sized groups. One
of them is assigned to the left half of the placement region,
and the other one to the right half. Partitioning is performed
by the Multi-level Fiduccia-Mattheyses ($\mathrm{MLFM}$)
heuristic \cite{PapMar07} to minimize connections between the two
groups of vertices (the net-cut objective). Each half is partitioned
again, but takes into account the connections to
the other half \cite{RoyAdyPapMar06}. At the large scale, ensuring the similar
sizes of bi-partitions corresponds to density constraints
and cut minimization corresponds to $\mathrm{HPWL}$ minimization.
When regions become small and contain $< 10$ vertices,
optimal positions can be found with respect to discrete
slot constraints by branch-and-bound \cite{CalKahMar00}. Balanced
hypergaph partitioning is NP-hard \cite{CGKMSP98}, but the $\mathrm{MLFM}$
heuristic takes $O((V + E)\log V)$ time. The entire min-cut
placement procedure takes $O((V + E)(\log V)^2)$ time and
can process hypergraphs with millions of vertices in several
hours.

A special case of interest is that of one-dimensional
placement. When all vertices have identical width and
none of them are fixed, one obtains the NP-hard \textsc{Minimum
Linear Arrangement} problem \cite{CGKMSP98} which can
be approximated in polynomial time within $O(\log V)$
and solved exactly for trees in $O(V^3)$ time as shown by
Yannakakis. The min-cut technique described above also
works well for the related NP-hard \textsc{Minimum-Cut
Linear Arrangement} problem \cite{CGKMSP98}.

\subsection{Nonlinear Optimization}
Quadratic and generic non-linear optimization may be
faster than linear programming, while reasonably approximating
the original formulation. The hypergraph is represented
by a weighted graph where $w_{ij}$ represents the
weight on the 2-pin edge connecting vertices $v_i$ and $v_j$ in
the weighted graph. When an edge is absent, $w_{i j} = 0$, and
in general $w_{ii} = -\sum_{i\neq j}w_{i j}$.

\subsubsection{Quadratic Placement} A quadratic placement ($x$-direction
only) is given by
\begin{equation}\label{eq:first}
\Phi(x) = \sum_{i,j}w_{ij}\left[(x_i-x_j)^2\right]=
\frac12\vec{x}^T\vec{Qx}+\vec{c}^T\vec{x}+\text{const.}
\end{equation}

The global minimum of $\Phi(x)$ is found by solving $\vec{Qx}+\vec{c} = 0$
which is a sparse, symmetric, positive-definite system of
linear equations (assuming $\ge 1$ fixed vertex), efficiently
solved to sufficient accuracy using any number of iterative
solvers. Quadratic placement may have different optima
depending on the model (clique or star) used to represent
hyperedges. However, for a $k$-pin hyperedge, if the
weight on the 2-pin edges introduced is set to $W_c$ in the
clique mode and $kW_c$ in the star model, then the models
are equivalent in quadratic placement \cite{RedCho06}.

\subsubsection{Linearized Quadratic Placement} Quadratic placement
can produce lower quality placements. To approximate the
linear objective, one can iteratively solve Eq.~(\ref{eq:first}) with $w_{i j} =
1/|x_i-x_j|$ computed at every iteration. Alternatively, one
can solve a single $\beta$-regularized optimization problem
given by $\Phi^\beta(x) = \min_x\sum_{i,j}w_{ij}\sqrt{x_i-x_j)^2+\beta}.\ \beta > 0$,
e.g., using a Primal-Dual Newton method with quadratic
convergence \cite{AlpChKaMaMu98}.

\subsubsection{Half-Perimeter Wirelength Placement} $\mathrm{HPWL}$ can be
provably approximated by strictly convex and differentiable
functions. For 2-pin hyperedges, $\beta$-regularization
can be used \cite{AlpChKaMaMu98}. For an $m$-pin hyperedge ($m \ge 3$), one
can rewrite $\mathrm{HPWL}$ as the maximum ($l_\infty$-norm) of all
$m(m-1)/2$ pairwise distances $|x_i - x_j|$ and approximate
the $l_\infty$-norm by the $l_p$-norm ($p$-th root of the sum of $p$-th
powers). This removes all non-differentiabilities except
at $0$ which is then removed with $\beta$-regularization. The resulting
$\mathrm{HPWL}$ approximation is given by
\begin{equation}\label{eq:second}
\mathrm{HPWL}_{p-\beta-\mathrm{reg}}(G_h)=\sum_{e_k\in E_h}\left(\sum_{i,j\in C_k}|x_i-x_j<^p+\beta\right)^{1/p}
\end{equation}
which overestimates $\mathrm{HPWL}$ with arbitrarily small relative
error as $p\to \infty$ and $\beta\to0$ \cite{RedCho06}. Alternatively, $\mathrm{HPWL}$ can
be approximated via the log-sum-exp formula given by
\begin{equation}\label{eq:third}
\mathrm{HPWL}_{\text{log-sum-exp}}(G_h)=
\alpha\sum_{e_k\in E_h}\left[\ln\left(\sum_{i\in C_k}\exp
\left(\frac{x_i}{\alpha}\right)\right)+\ln\left(\sum_{v_i\in C_k}
\exp\left(\frac{-x_i}{\alpha}\right)\right)\right]
\end{equation}
where $\alpha > 0$ is a smoothing parameter \cite{KhaWan05}. Both approximations
can be optimized using conjugate gradient methods.

\subsection{Analytic Techniques for Target Density Constraints}
The target density constraints are non-differentiable and
are typically handled by approximation.

\subsubsection{Force-Based Spreading} The key idea is to add constant
forces f that pull vertices always from overlaps, and recompute
the forces over multiple iterations to reflect changes
in vertex distribution. For quadratic placement, the new
optimality conditions are $\vec{Qx} + \vec{c} + \vec{f} = \vec{0}$ \cite{KenVor06}. The constant
force can perturb a placement in any number of ways to
satisfy the target density constraints. The force $\vec{f}$ is computed
using a discrete version of Poisson's equation.

\subsubsection{Fixed-Point Spreading} A fixed point $f$ is a pseudovertex
with zero area, fixed at $(x_f,y_f)$, and connected to
one vertex $H(f)$ in the hypergraph through the use of
a pseudo-edge with weight $w_{f,H(f)}$. Quadratic placement
with fixed points is given by $\Phi(x) = \sum_{i,j} w{i,j}(x_i-x_j)^2 +
\sum_fw_{f,H(f)}(x_{H(f)}-x_f)^2$.
Each each fixed point $F$f introduces
a quadratic term $w_{f,H(f)}(x_{H(f)}-x_f)^2$. By manipulating
the positions of fixed points, one can perturb a placement
to satisfy the target density constraints. Compared
to constant forces, fixed points improve the controllability
and stability of placement iterations \cite{HuMar05}.

\subsubsection{Generalized Force-Directed Spreading} The Helmholtz
equation models a diffusion process and makes it ideal for
spreading vertices \cite{ChaConSze05}. The Helmholtz equation is given by
\begin{gather}\label{eq:four}
\frac{\partial^2\phi(x,y)}{\partial x^2}
\frac{\partial^2\phi(x,y)}{\partial y^2}-
\epsilon\phi(x,y)=D(x,y),\\
(x,y)\in R\frac{\partial \phi}{\partial v}= 0,
(x,y) \text{ on the boundary of } R\nonumber
\end{gather}
where $\epsilon>0, v$ is an outer unit normal, $R$ represents the
fixed outline, and $D(x,y)$ represents the continuous density
function. The boundary conditions, $\partial\Phi/\partial v = 0$, specify
that forces pointing outside of the fixed outline be set
to zero -- this is a key difference with the Poisson method
which assumes that forces become zero at infinity. The
value $\Phi_{ij}$ at the center of each bin $B_{ij}$ is found by discretization
of Eq. (\ref{eq:four}) using finite differences. The density constraints
are replaced by $\Phi_{ij}=\hat K,\forall b_{ij}\in B$ where $\hat K$ is
a scaled representative of the density target $K$. Wirelength
minimization subject to the smoothed density constraints
can be solved via Uzawa's algorithm. For quadratic wirelength,
this algorithm is a generalization of force-based
spreading.

\subsubsection{Potential Function Spreading} Target density constraints
can also be satisfied via a penalty function. The
area assigned to bin $B_{ij}$ by vertex $v_i$ is represented by
$\hbox{Potential}(v_i,B_{ij})$ which is a bell-shaped function. The use
of piecewise quadratic functions make the potential function
non-convex, but smooth and differentiable \cite{KhaWan05}. The
penalty term given by
\begin{equation}\label{eq:fifth}
\text{Penalty} = \sum_{B_{ij}\in B}\left(\sum_{v_i\in V_h}\hbox{Potential}(v_i,B_{ij})-K\right)^2
\end{equation}
can be combined with a wirelength approximation to arrive
at an unconstrained optimization problem which is
solved using an efficient conjugate gradient method \cite{KhaWan05}.

\Applic

Practical applications involve more sophisticated interconnect
objectives, such as circuit delay, routing congestion,
power dissipation, power density, and maximum thermal gradient.
The above techniques are adapted to
handle multi-objective optimization. Many such extensions
are based on heuristic assignment of net weights that
encourage the shortening of some (e.g., timing-critical
and frequently-switching) connections at the expense of
other connections. To moderate routing congestion, predictive
congestion maps are used to decrease the maximal
density constraint for placement in congested regions. Another
application is in physical synthesis, where incremental
placement is used to evaluate changes in circuit topology.

\OpenProb

None is reported.

\ExpRes

Circuit placement has been actively studied for the past
30 years and a wealth of experimental results are reported
throughout the literature. A 2003 result demonstrated that
placement tools could produce results as much as $1:41\times$
to $2:09\times$ known optimal wirelengths on average (advances
have been made since this study). A 2005 placement contest
found that a set of tools produced placements with
wirelengths that differed by as much as $1:84\times$ on average.
A 2006 placement contest found that a set of tools produced
placements that differed by as much as $1:39\times$ on average
when the objective was the simultaneous minimization
of wirelength, routability and run time. Placement run
times range from minutes for smaller instances to hours
for larger instances, with several millions of variables.

\DataSets

Benchmarks include the ICCAD '04 suite
(\url{http://vlsicad.eecs.umich.edu/BK/ICCAD04bench/}), the ISPD '05
suite (\url{http://www.sigda.org/ispd2005/contest.htm}) and the ISPD '06
suite (\url{http://www.sigda.org/ispd2006/contest.htm}). Instances in
these benchmark suites contain between 10K to 2.5M placeable objects.
Other common suites can be found, including large-scale placement
instances problems with known optimal solutions
(\url{http://cadlab.cs.ucla.edu/~pubbench}).

\CrossRef
\href{http://link.springer.com/referenceworkentry/10.1007%2F978-0-387-30162-4_291}
{Performance-Driven Clustering}\\
Energy Minimization in VLSI Circuits

%% BibTeX users please use
%\bibliographystyle{spbasic}
%\bibliography{<name of bib_database>}
%
%\end{document}
%
% Non-BibTeX users please use
%\begin{thebibliography}{8.}
%
% and use \bibitem to create references.
%
% Use the following syntax and markup for your references
%


%% Complete Book
%\bibitem{book} Faber K (1997) Biotransformations in organic
%chemistry, 3rd edn. Springer, Berlin Heidelberg New York
%
%% Contributions, Chapters or Sections in Books
%\bibitem{contribution} Brown FR, Jackson D (1987) Analytical chemistry.
%In: Whie FC, Red SA (eds) Modern chemistry. Springer, Berlin Heidelberg
%New York, p 220
%
%% Journal
%\bibitem{journal}
%Fuchigami T (1994) Top Curr Chem 170:25
%
%% Other Possibilities
%\bibitem{Other} Tanzawa H (1986) First Japan-US workshop on Biomedical
%Polymer Science. Kyoto, Japan
%
%\end{thebibliography}
%
%%%%%%%%%%%%%%%%%%%%%%%%%%%%%%%%%%%%%%%%%%%%%%%%%%%%%%%%%%%%%%%%%%%%%%%
%
%\end{document}

% here a faked reference section follows working with the natbib-system
% allowing for author-year citation without automated BibTeX usage

% Non-BibTeX users please use

\begin{thebibliography}{12}
%
% and use \bibitem to create references. Consult the Instructions
% for authors for reference list style.

\bibitem[{{Alpert} et~al.(1998){Alpert}, {Chan}, {Kahng}, {Markov},
{Mulet}}]{AlpChKaMaMu98} {Alpert} CJ,
{Chan} T, {Kahng} AB, {Markov} IL, {Mulet} P (1998) Faster
minimization of linear wirelength for global placement. IEEE
Trans. CAD 17(1), 3-13

\bibitem[{{Caldwell} et~al.(2000){Caldwell}, {Kahng},
{Markov}}]{CalKahMar00} {Caldwell} AE, {Kahng} AB, {Markov} IL (2000)
Optimal partitioners and end-case placers for standard-cell layout.
IEEE Trans. CAD 19(11), 1304-1314

\bibitem[{{Chan} et~al.(2005){Chan}, {Cong}, {Sze}}]{ChaConSze05}
{Chan} T, {Cong} J, {Sze} K (2005) Multilevel generalized force-directed
method for circuit placement. Proc. Intl. Symp. Physical Design.
ACM Press, San Francisco, 3-5 Apr 2005. pp. 185-192

\bibitem[{{Ausiello} et~al.(1998)
{Ausiello}, {Crescenzi}, {Gambosi}, {Kann}, {Marchetti-Spaccamela},
{Protasi}}] {CGKMSP98} {Ausiello} G, {Crescenzi} P, {Gambosi} G,
{Kann} V, {Marchetti-Spaccamela} A, {Protasi} M (1998) Complexity and
Approximation: Combinatorial optimization problems and their
approximability properties. Springer

\bibitem[{{Hu} et~al.(2005){Hu}, {Marek-Sadowska}}]{HuMar05}
{Hu} B, {Marek-Sadowska} M (2005) Multilevel fixed-point-additionbased
VLSI placement. IEEE Trans. CAD 24(8), 1188-1203

\bibitem[{{Kahng} et~al.(2005){Khang}, {Wang}}]{KhaWan05}
{Kahng} AB, {Wang} Q (2005) Implementation and extensibility of an
analytic placer. IEEE Trans. CAD 24(5), 734-747

\bibitem[{{Kennings} et~al.(2002){Kennings}, {Markov}}]{KenMar02}
{Kennings} A, {Markov} IL (2002) Smoothingmax-terms and analytical
minimization of half-perimeter wirelength. VLSI Design 14(3),
229-237

\bibitem[{{Kennings} et~al.(2006){Kennings}, {Vorwerk}}]{KenVor06}
{Kennings} A, {Vorwerk} K (2006) Force-directed methods for generic
placement. IEEE Trans. CAD 25(10), 2076-2087

\bibitem[{{Papa} et~al.(2007){Papa}, {Markov}}]{PapMar07}
{Papa} DA, {Markov} IL (2007) Hypergraph partitioning and clustering.
In: Gonzalez, T. (ed.) Handbook of algorithms. Taylor \&
Francis Group, Boca Raton, CRC Press, pp. 61-1

\bibitem[{{Reda} et~al.(2006){Reda}, {Chowdhary}}]{RedCho06} {Reda} S,
{Chowdhary} A (2006) Effective linear programming based
placement methods. In: ACM Press, San Jose, 9-12 Apr

\bibitem[{{Roy} et~al.(2006){Roy}, {Adya}, {Papa}, {Markov}}]{RoyAdyPapMar06}
{Roy} JA, {Adya} SN, {Papa} DA, {Markov} IL(2006) Min-cut floorplacement.
IEEE Trans. CAD 25(7), 1313-1326

\bibitem[{{Tang} et~al.(2005){Tang}, {Tian}, {Wong}}]{TanTiaWon05}
{Tang} X, {Tian} R, {Wong} MDF (2005) Optimal redistribution of white
space for wirelength minimization. In: Tang, T.-A. (ed.) Proc.
Asia South Pac. Design Autom. Conf., ACM Press, 18-21 Jan
2005, Shanghai. pp. 412-417

\end{thebibliography}

\end{document}
% end of file "example.tex"


