\section{Introduction}
\label{sec:intro}

Planar graphs have been a natural focus of study for algorithms research for decades, both because they accurately model many real-world networks, and because they often admit simpler and/or more efficient algorithms for many problems than general graphs.  Most planar-graph algorithms either apply immediately or have been quickly generalized to larger families of graphs, such as graphs of higher genus, graphs with forbidden minors, or graphs with small separators.  Examples include minimum spanning trees \cite{p-omst-99, m-tltam-04}; single-source and multiple-source shortest paths \cite{cc-msspg-07, fr-pgnwe-06, hkrs-fspap-97, k-msspp-05, kmw-spdpg-09, lrt-gnd-79, tm-spltm-09}; graph and subgraph isomorphism \cite{g-itegd-00, hw-ltaip-74, m-itgbg-80, e-sipgr-99, e-dtmcg-00}; and approximation algorithms for the traveling salesman problem, Steiner trees, and other NP-hard problems~\cite{bdt-ptass-08, bkk-ptass-07, bkk-stpg-07, dhm-aacd-07, e-dtmcg-00}.

The classical minimum cut problem and its dual, the maximum flow problem, are stark exceptions to this general pattern.  Flows and cuts were introduced in the 1950s as tools for studying transportation networks, which are naturally modeled as planar graphs \cite{hr-fmern-55}.  Ford and Fulkerson's seminal paper \cite{ff-mfn-56} includes an algorithm to compute maximum flows in planar networks where the source and target lie on the same face.  A long series of results eventually led to planar minimum-cut algorithms that run in $O(n\log n)$ time, first for undirected graphs \cite{r-mstcp-83, hj-oamfu-85, f-faspp-87} and later for directed
graphs \cite{jk-mcdpn-92, hkrs-fspap-97}.
Strangely, however, almost nothing was known about computing flows and cuts in generalizations of planar graphs until very recently~\cite{cen-hfcc-12}.
Even for graphs embedded on the torus, the fastest known minimum-cut algorithms require first computing a maximum flow, a process which takes~$O(n^{3/2})$ time unless edge capacities are sufficiently small integers~\cite{cen-hfcc-12}.

This paper describes the first algorithms to compute minimum cuts in surface-embedded graphs of fixed genus in near-linear time when edges are given real valued capacities.
Two of these algorithms are designed to solve the minimum $(s,t)$-cut problem; the remaining algorithm solves the global minimum cut problem.
Before describing our results in detail, we first review several related results; technical terms are defined in Section \ref{sec:prelims}.

\fakeparagraph{Planar minimum cuts.}
\note{TODO(Kyle): Make sure we do not use ``minimum cut'' when we specifically mean ``minimum $(s,t)$-cut''.}
Recall that for any two vertices $s$ and~$t$ in a graph $G$, an \EMPH{$(s,t)$-cut} is a subset of the edges of $G$ that intersects every path from $s$ to $t$.  A \emph{minimum} $(s,t)$-cut is an $(s,t)$-cut of minimum size, or minimum total weight if the edges of $G$ are weighted.

Itai and Shiloach \cite{is-mfpn-79} observed that the minimum $(s,t)$-cut in a planar graph~$G$ is dual to the minimum-cost cycle that separates faces $s^*$ and $t^*$ in the dual graph $G^*$.  They also observed that this separating cycle intersects any shortest path from a vertex of $s^*$ to a vertex of $t^*$ exactly once.  Thus, one can compute the minimum cut by cutting the dual graph $G^*$ along a shortest path~$\pi$ from $s^*$ to~$t^*$; duplicating every vertex and edge of $\pi$; and then computing, for each vertex $u$ of $\pi$, the shortest path between the two copies of $u$ in the resulting planar graph.  Applying Dijkstra's shortest-path algorithm at each vertex of~$\pi$ immediately yields a running time of $O(n^2\log n)$.

Reif \cite{r-mstcp-83} improved the running time of this algorithm to $O(n\log^2 n)$ using a divide-and-conquer strategy.  Reif's algorithm was extended by Hassin and Johnson to compute the actual maximum flow in $O(n\log n)$ additional time, using a carefully structured dual shortest-path computation \cite{hj-oamfu-85}.  Frederickson~\cite{f-faspp-87} subsequently improved the running time of Reif's algorithm to $O(n\log n)$ using a balanced separator decomposition to speed up the shortest-path computations.  Janiga and Koubek~\cite{jk-mcdpn-92} attempted to adapt Reif's $O(n\log^2 n)$-time algorithm to directed planar graphs\footnote{Unfortunately, their minimum cut algorithm has a subtle
error~\cite{kn-mcupg-11} which may lead to an incorrect result when the
minimum $t,s$-cut is smaller than the minimum $s,t$-cut.}.

Henzinger \etal~\cite{hkrs-fspap-97} generalized Frederickson's technique to obtain an $O(n)$-time planar shortest-path algorithm; using this algorithm in place of Dijkstra's algorithm improves the running times of both Reif's and Janiga and Koubek's algorithms to $O(n\log n)$.  The same improvement can also be obtained using more recent multiple-source shortest path algorithms by Klein~\cite{k-msspp-05} and Cabello and Chambers \cite{cc-msspg-07}.

Minimum $(s,t)$-cuts in directed planar graphs can also be obtained in $O(n\log n)$ time using the planar maximum-flow algorithms of Weihe \cite{w-mstfp-97} (if the graph satisfies certain connectivity restrictions) and Borradaile and Klein \cite{b-epnfc-08, bk-tamfd-06, bk-amfdp-09}.

A \EMPH{cut} (without specified $s$ and $t$) is a subset of edges of $G$ that separate $G$ into two non-empty sets of vertices.
A \emph{minimum} cut is a cut of minimum size, or minimum total weight if the edges of $G$ are weighted.
To differentiate between minimum $(s,t)$-cuts and minimum cuts, we sometimes refer to the latter as global minimum cuts.
Chalermsook, Fakcharoenphol, and Nanongkai~\cite{cfn-dnlta-04} gave the first algorithm for computing global minimum cuts that relies on planarity. Their algorithm runs in~$O(n \log^2 n)$ time. Their algorithm was later improved by \L\c{a}cki and Sankowski~\cite{ls-mcsc-11} who achieved an $O(n \log \log n)$ running time.

\fakeparagraph{Generalizations of planar graphs.}
Surprisingly little is known about the complexity of computing maximum flows or minimum cuts in generalizations of planar graphs.  In particular, we know of no algorithm to compute minimum cuts in non-planar graphs that does not first compute a maximum flow.

By combining a technique of Miller and Naor \cite{mn-fpgms-95} with the planar directed flow algorithm of Borradaile and Klein \cite{b-epnfc-08, bk-tamfd-06, bk-amfdp-09}, one can compute maximum (single-commodity) flows in a planar graph with $k$ sources and sinks in $O(k^2 n\log n)$ time.  A recent algorithm of Hochstein and Weihe \cite{hw-mstfkc-07} computes a maximum flow in a planar graph with $k$ additional edges in $O(k^3n\log n)$ time, using a clever simulation of Goldberg and Tarjan's push-relabel algorithm~\cite{gt-namfp-88}.

Imai and Iwano \cite{ii-espap-90} describe a max-flow algorithm that applies to graphs of positive genus, but not to arbitrary sparse graphs.
Their algorithm computes minimum-cost flows in graphs with small balanced separators, using a combination of nested dissection \cite{lrt-gnd-79, pr-fepss-93}, interior-point methods~\cite{v-slpfm-89}, and fast matrix multiplication.
Their algorithm can be adapted to compute maximum flows (and therefore minimum cuts) in any graph of constant genus in time $O(n^{1.595}\log C)$, where $C$ is the sum of integer edge weights.
However, this algorithm is slower than more recent and more general algorithms \cite{gr-bfdb-98}.
Chambers, Erickson, and Nayyeri~\cite{cen-hfcc-12} describe maximum flow algorithms that are tailored specifically for graphs of constant genus.
Given a graph embedded on a surface of genus~$g$, their algorithms can compute a maximum flow in time $O(g^8 n \log^2 n \log^2 C)$ where $C$ is the sum of integer edge weights and time $g^{O(g)}n^{3/2}$ when edge weights are arbitrary positive real numbers.  Their key insight is that it suffices to optimize the relative \emph{real} homology class of the flow, rather than directly optimizing the flow itself.

Euler's formula implies that a simple $n$-vertex graph embedded on a surface of genus $O(n)$ has at most $O(n)$ edges.
The fastest known combinatorial maximum-flow algorithm for sparse graphs, due to Orlin~\cite{o-mfotl-13}, runs in time $O(n^2 / \log n)$.
The fastest algorithm known for small integer capacities, due to Goldberg and Rao \cite{gr-bfdb-98}, runs in time $O(n^{3/2}\log n\log U)$, where $U$ is an upper bound on the integer edge capacities.
\note{Kyle: We use weight almost everywhere except for a few times when we use capacity (or cost! —Jeff).
I think we should stick with weight unless we're talking about flows, or try harder to use only one or the other.}
Finally, the fastest known algorithm for computing global minimum cuts for sparse graphs, due to Karger~\cite{k-mcnlt-00}, runs in~$O(n \log^3 n)$ time.

For further background on maximum flows, minimum cuts, and related problems, we refer the reader to monographs by Ahuja \etal\ \cite{amo-nftaa-93} and Schrijver \cite{s-cape-03}.

\fakeparagraph{Short interesting cycles.}
Many different problems, such as finding approximate traveling salesman tours  \cite{dhm-aacd-07} and Steiner trees~\cite{bdt-ptass-08} in surface embedded graphs, embedding high-genus graphs into the plane with low distortion \cite{is-pebgg-07} or with few crossings \cite{kr-ccnlt-07}, and feature detection and simplification in meshes \cite{gw-tnr-01,dlsc-cgaht-08}, require algorithms to find short but topologically nontrivial cycles.  These problems are closely connected to that of finding cuts on surface-embedded graphs, since any minimum cut will form a small set of cycles separating the vertices.

Thomassen described the first efficient algorithm to find the shortest non-contractible or non-separating cycle \cite{t-egnsn-90, mt-gs-01}.  After many intermediate improvements \cite{c-mdpg-06, cm-fsnsn-07, eh-ocsd-04, k-csnco-06,cc-msspg-07,cce-msspe-13,insw-iamcmf-11}, Fox~\cite{f-sntcd-13} described the fastest algorithm currently known for this problem, which runs in $2^{O(g)} n \log \log n$ time.
Splitting cycles are non-contractible and separating; finding the shortest such cycle is {NP}-hard, although there is an $O(n \log \log n)$-time algorithm for graphs of any fixed genus \cite{ccelw-scsih-06, ccelw-scsih-08,insw-iamcmf-11}.  Colin de Verdi\`ere and Erickson~\cite{ce-tspcs-06} prove that the shortest path or cycle in a given homotopy class can be computed in polynomial time, improving earlier results of Colin de Verdi\`ere and Lazarus \cite{c-rcds-03, cl-oslos-05, cl-opdsh-07}.  Erickson and Whittlesey describe a greedy algorithm to find a minimum-length set of cycles that generate the first homology group of a surface \cite{ew-gohhg-05}.
Finally, Cabello, Colin de Verdi\`ere, and Lazarus~\cite{ccl-fsncd-10} described the first efficient algorithms to find the shortest \emph{directed} non-contractible or non-separating cycle in a directed surface embedded graph.
Erickson~\cite{e-sncds-11} and Fox~\cite{f-sntcd-13} describe improvements to their algorithm that find the shortest non-separating cycle in~$O(g^2 n \log n)$ time and the shortest non-contractible cycle in~$O(g^3 n \log n)$ time respectively.

Several practical heuristics have been developed for finding short cycles that work well in practice, although they have no theoretical guarantees.  For example, Guskov and Wood \cite{gw-tnr-01} describe an algorithm to find and remove intersecting pairs of short non-contractible cycles (`topological noise') from surface meshes.  Zomorodian and Carlsson \cite{zc-lh-07} define the \emph{localized} homology of an arbitrary topological case with respect to a subspace covering; they also describe algorithms to compute localized homology generators of simplicial complexes using persistent homology; however, they do not describe how to compute covers that would lead to cycles of minimum size.  Chen and Friedman \cite{cf-qhc2-07, cf-qhc-08} describe a polynomial-time algorithm to compute a cycle of minimum \emph{radius} in a given homology class, in an arbitrary edge-weighted simplicial complex; however, the length of this cycle could be arbitrarily longer than optimal.

Dey \etal~\cite{dls-chtl-07, dlsc-cgaht-08} describe algorithms to find short \emph{handle} and \emph{tunnel} cycles in surface meshes embedded in $\Real^3$.  Any embedded surface subdivides $\Real^3$ into an inner handlebody $I$ and an outer handlebody $O$; handle cycles are null-homologous in $I$, while tunnel cycles are null-homologous in~$O$.  The algorithm of Dey \etal\ finds $g$ short handle cycles and $g$ short tunnel cycles that collectively generate the first homology group of the input surface.  Their algorithm uses several heuristics to reduce the length of the output cycles, in part because no algorithm is known to find the shortest cycle in a given homology class.

\subsection{New results and organization}

\textcolor{BrickRed}{\textsf{\textbf{High-level overview:}
Our new results are based on a simple topological characterization.  By definition, a set $C$ of edges defines an $(s,t)$-cut in a graph $G$ if and only if deleting those edges leaves a disconnected graph, with $s$ and $t$ in different components.  If $G$ is embedded on a surface, than deleting the corresponding edges $C^*$ in the dual graph $G^*$ 
}}

The input to our first algorithm is an undirected edge-weighted graph~$G$ embedded on an orientable surface of genus~$g$.  Given vertices $s$ and $t$, our algorithm computes a minimum-weight $(s,t)$-cut in $g^{O(g)}n\log \log n$ time.
For any fixed positive genus, this result improves the best previous time bound of Chambers, Erickson, and Nayyeri~\cite{cen-hfcc-12} by a factor of $\min\set{ \sqrt{n} / \log \log n, \log^2 n / \log \log n \log^2 C}$ (compared to minimum cut algorithms for general sparse graphs, the improvement is $\min\set{n / ( \log n \log \log n),\allowbreak
 \sqrt{n} \log n / \log \log n \log C}$).
The bottleneck for our algorithm is the time it takes to compute a minimum weight $(s,t)$-cut in a \emph{planar} embedded graph.
For any fixed positive genus, our algorithm's running time matches the best known for planar graphs, and will continue to do so as algorithms for planar graphs improve.

Our minimum-cut algorithm is a special case of a more general algorithm to compute a minimum-cost \emph{subgraph} in a given $\Z_2$-homology class.  Even when the homology class is specified by a simple cycle, the output representative may be the union of several cycles; see Figure \ref{F:homology2}.  For surfaces with genus $g$ and $b$ boundary components, our algorithm runs in $(g+b)^{O(g+b)}n\log\log n$ time.
We give this algorithm in Section~\ref{sec:crossing}.
In Section~\ref{sec:hardness}, we show that this more general problem is strongly NP-hard, even if the input homology class is specified by a simple cycle; thus, the exponential dependence on the topology of the surface is unavoidable unless {P}={NP}.  We are not aware of any previous algorithmic results for our more general problem.

The input to our second algorithm is a \emph{directed} edge-weighted graph~$G$ embedded on an orientable surface of genus~$g$.
\note{Kyle: Does the surface need to be orientable?}
This algorithm computes a shortest directed \emph{cycle} in a specified $\Z_2$-homology class in $2^{O(g)}n\log n$ time using different techniques from the algorithm above.
We are unaware of any previously published algorithm for this specific problem; however, an algorithm of Chambers \etal~\cite{ccelw-scsih-08} can be modified to find shortest homologous cycles in \emph{undirected} graphs in $g^{O(g)} n\log n \log n$ time.
This problem can be shown to be {NP}-hard, even for undirected graphs, by a reduction from the traveling salesman problem in grid graphs~\cite{ccelw-scsih-08}.
We also show how to extend this algorithm to compute a minimum-weight $(s,t)$-cut in $2^{O(g)}n\log n$ time in an undirected graph.
These algorithms are described in Section~\ref{sec:homcover}.

Our final algorithm takes as input an undirected edge-weighted graph~$G$ embedded on an orientable surface of genus~$g$.
This algorithm computes a global minimum cut in $g^{O(g)}n\log \log n$ time.
Again, the bottleneck for this algorithm is the time it takes to compute a global minimum cut in a \emph{planar} embedded graph.
For any fixed positive genus, our algorithm's running time matches the best known for planar graphs, and will continue to do so as algorithms for planar graphs improve.
We present the global minimum cut algorithm in Section~\ref{sec:global}.

As in the case of computing minimum-weight $(s,t)$-cuts, our algorithm for global minimum cuts depends upon our more general algorithm for minimum-cost subgraphs in given $\Z_2$-homology classes.
The more general problem is NP-hard, but of course, both versions of the original minimum-cut problem are not NP-hard!
We conjecture that minimum-weight $(s,t)$-cuts and global minimum cuts can be computed in time $O(g^c n\log \log n)$ for some small constant $c$.

%As mentioned, Chambers, Erickson, and Nayyeri~\cite{cen-hfcc-12} describe algorithms to compute a maximum flow in surface embedded graphs, using very different techniques from this paper.  Specifically, given a directed graph embedded on an orientable surface of genus~$g$, with two specified vertices $s$ and $t$, they can compute a maximum $(s,t)$-flow in $O(g^8 n\log^2 n\log^2 C)$ time for integer capacities that sum to $C$, or in $g^{O(g)}n^{3/2}$ time for real capacities.  Their key insight is that it suffices to optimize the relative \emph{real} homology class of the flow, rather than directly optimizing the flow itself.
%\note{Kyle: These no longer look like sister papers, but I like mentioning the real homology part. Where should we put this paragraph?}
