\documentclass[natbib]{svcyclop}
\usepackage{graphicx} 
\usepackage{amsfonts,amsmath}
\usepackage{hyperref}
\usepackage{url}
\urlstyle{tt}

\usepackage[utf8x]{inputenc}	% Understand Unicode in source file
\usepackage[T1]{fontenc}	% OT1 does not support ogonek (\k) in Łącki = \L\k{a}cki
\usepackage{microtype}		% Oh, yes, thank GOD.

% ----------------------------------------------------------------------
\begin{document}

\title{Global Minimum Cuts in Surface Embedded Graphs}

% - - - - - - - - - - - - - - - - - - - - - - - - - - - - - - - - - - - - - - - - 

\author{
	Erin W. Chambers\inst{1}
\and
	Jeff Erickson\inst{2}
\and
	Kyle Fox\inst{3}
\and
	Amir Nayyeri\inst{4}}

% - - - - - - - - - - - - - - - - - - - - - - - - - - - - - - - - - - - - - - - - 

\institute{
	Department of Computer Science and Mathematics,
	Saint Louis University, St. Louis, MO, USA
\and
	Department of Computer Science,
	University of Illinois, Urbana, IL, USA
\and
	Institute for Computational and Experimental Research in Mathematics,
	Brown University, Providence, RI, USA
\and
	Department of Electrical Engineering and Computer Science,
	Oregon State University, Corvallis, OR, USA}

% - - - - - - - - - - - - - - - - - - - - - - - - - - - - - - - - - - - - - - - - 

\sumoriwork{2009; Chambers, Erickson, Nayyeri\\
2011; Erickson, Nayyeri\\
2012; Erickson, Fox, Nayyeri}

% - - - - - - - - - - - - - - - - - - - - - - - - - - - - - - - - - - - - - - - - 

\keywords{Topological graph theory;
Graph embedding;
Minimum cuts; 
Homology;
Covering spaces;
Fixed-parameter tractability}

\maketitle

% ----------------------------------------------------------------------

\ProbDef

Given a graph $G$ in which every edge has a non-negative capacity, the goal of the minimum cut problem is to find a subset of edges of $G$ with minimum total capacity whose deletion disconnects~$G$.  The closely related minimum $(s,t)$-cut problem further requires two specific vertices $s$ and $t$ to be separated by the deleted edges.  Minimum cuts and their generalizations play a central role in divide-and-conquer and network optimization algorithms.

The fastest algorithms known for computing minimum cuts in arbitrary graphs run in roughly $O(mn)$ time for graphs with $n$ vertices and $m$ edges.  However, even faster algorithms are known for graphs with additional topological structure.  This article sketches algorithms to compute minimum cuts in near-linear time when the input graph can be drawn on a surface with bounded genus---informally, a sphere with a bounded number of handles. 

\subsubsection{Problem 1 (Minimum (s,t)-Cut)}
{\itshape
\textsc{Input}: An undirected graph $G = (V, E)$ embedded on an orientable surface of genus $g$, a non-negative capacity function $c\colon E \to \mathbb{R}$, and two vertices $s$ and $t$. \textsc{Output}:~A minimum-capacity $(s,t)$-cut in $G$.}

\subsubsection{Problem 2 (Global Minimum Cut)}
{\itshape
\textsc{Input}: An undirected graph $G = (V, E)$ embedded on an orientable surface of genus $g$ and a non-negative capacity function $c\colon E \to \mathbb{R}$.  \textsc{Output}: A minimum-capacity cut in $G$.}



% ----------------------------------------------------------------------

\KeyRes

\subsection{Topological Background}

A surface is a compact space in which each point has a neighborhood homeomorphic to either the plane or a closed halfplane.
Points with halfplane neighborhoods comprise the boundary of the surface, which is the union of disjoint simple cycles.
The genus is the maximum number of disjoint simple cycles whose deletion leaves the surface connected.
A surface is orientable if it does not contain a Möbius band.
An embedding is a drawing of a graph on a surface, with vertices drawn as distinct points and edges as simple interior-disjoint paths, whose complement is a collection of disjoint open disks called the faces of the embedding.

An even subgraph of $G$ is a subgraph in which every vertex has even degree; each component of an even subgraph is Eulerian.
Two even subgraphs of an embedded graph~$G$ are $\mathbb{Z}_2$-homologous, or in the same $\mathbb{Z}_2$-homology class, if their symmetric difference is the boundary of a subset of the surface.  If $G$ is embedded on a surface of genus $g$ with $b>0$ boundary cycles, the even subgraphs of $G$ fall into $2^{2g+b-1}$ $\mathbb{Z}_2$-homology classes.
An even subgraph of $G$ is $\mathbb{Z}_2$-minimal if it has minimum total cost within its $\mathbb{Z}_2$-homology class.
Each component of a $\mathbb{Z}_2$-minimal even subgraph is itself $\mathbb{Z}_2$-minimal.

Every embedded graph $G$ has a dual graph $G^*$, embedded on the same surface, whose vertices correspond to faces of $G$, and whose edges correspond to pairs of faces that share an edge in $G$.
The cost of a dual edge in $G^*$ is the capacity of the corresponding primal edge in~$G$.  

Duality maps cuts to certain sets of cycles and vice versa.  For example, the minimum-capacity $(s,t)$-cut in any planar graph $G$ is dual to the minimum-cost  cycle in~$G^*$ that separates the dual faces $s^*$ and $t^*$.  If we remove $s^*$ and $t^*$ from the sphere, the dual of the minimum cut is the shortest generating cycle of the resulting annulus~\cite{is-mfpn-79, r-mstcp-83}.  More generally, let $X$ denote the set of edges that cross some minimum $(s,t)$-cut in an embedded graph $G$, and let~$X^*$ denote the corresponding subgraph of the dual graph~$G^*$.  Then~$X^*$ is a minimum-cost even subgraph of $G^*$ that separates $s^*$ and $t^*$.  If we remove  $s^*$ and $t^*$ from the surface, $X^*$ becomes a $\mathbb{Z}_2$-minimal subgraph homologous with the boundary of $s^*$.

\subsection{Crossing Sequences}

Our first algorithm \cite{cen-mcshc-09} reduces computing a minimum $(s,t)$-cut in a graph embedded on a genus-$g$ surface to $g^{O(g)}$ instances of the planar minimum-cut problem.

The algorithm begins by constructing a collection $A$ of $2g+1$ paths in $G^*$, called a greedy system of arcs, with three important properties.  First, the endpoints of each path are incident to the boundary faces $s^*$ and $t^*$.  Second, each path is composed of two shortest paths plus at most one additional edge.  Finally, the complement $\Sigma\setminus A$ of the paths is a topological open disk.  A greedy system of arcs can be computed in $O(gn)$ time \cite{ccelw-scsih-08,en-mcsnc-11}.

We regard each component of $X^*$ as a closed walk, and we enumerate all possible sequences of crossings between the components of $X^*$ with the arcs in $A$.
The components of any $\mathbb{Z}_2$-minimal even subgraph cross any shortest path, and therefore any arc in~$A$, at most $O(g)$ times.  It follows that we need to consider at most~$g^{O(g)}$ crossing sequences, each of length at most $O(g^2)$.
Following Kutz \cite{k-csnco-06}, the shortest closed walk with a given crossing sequence is the shortest generating cycle in an annulus obtained by gluing together $O(g^2)$ copies of the disk $\Sigma\setminus A$, which can be computed in $O(g^2n\log\log n)$ time using the planar minimum-cut algorithm of Italiano \etal~\cite{insw-iamcmf-11}.
The overall running time of this algorithm is $g^{O(g)}n\log\log n$.

Surprisingly, a reduction from \textsc{MaxCut} implies that finding the minimum-cost even subgraph in an arbitrary $\mathbb{Z}_2$-homology class is NP-hard.  Different reductions imply that it is NP-hard to find the minimum-cost \emph{closed walk} \cite{ccelw-scsih-08} or \emph{simple cycle}~\cite{c-fscss-10} in a given $\mathbb{Z}_2$-homology class.

\subsection{$\mathbb{Z}_2$-Homology Cover}

Our second algorithm \cite{en-mcsnc-11} finds the minimum-cost closed walks in $G^*$ in every $\mathbb{Z}_2$-homology class by searching a certain covering space and then assembles $X^*$ from these closed walks via dynamic programming.

As in our first algorithm, we first compute a greedy system of arcs $A$.
The homology class of any cycle $\gamma$ is determined by the parity of the number of crossings of $\gamma$ with each arc in $A$.
Each arc $\alpha_i\in A$ appears as two paths $\alpha_i^+$ and $\alpha_i^-$ on the boundary of the disk $D = \Sigma\setminus A$.

We then construct a new surface $\overline\Sigma$, called the \emph{$\mathbb{Z}_2$-homology cover} of $\Sigma$, by gluing together several copies of $D$ as follows.   We associate each homology class $h\in\mathbb{Z}_2^{2g+1}$ with a vector of $2g+1$ bits.  Let $h \land i$ denote the bit vector obtained from $h$ by flipping its $i$th bit.  For each bit vector $h$, we construct a copy $D_h$ of $D$; let $\alpha^+_{i,h}$ and $\alpha^-_{i,h}$ denote the copies of $\alpha_i^+$ and $\alpha_i^-$ on the boundary of $D_h$.  Finally, we construct $\overline\Sigma$ by identifying the paths $\alpha^+_{i,h}$ and $\alpha^-_{i,h \land i}$ for each homology class $h$ and index $i$.%  The resulting  surface~$\overline{\Sigma}$ has genus $2^{O(g)}$.

This construction also yields a graph $\overline{G}$ embedded in $\overline\Sigma$, with $2^{2g+1}$ vertices $v_h$ and edges $e_h$ for each each vertex $v$ and edge $e$ of $G^*$.
Each edge $e_h$ of $\overline{G}$ inherits the cost of the corresponding edge $e$ in $G^*$.  
Any walk in $\overline{G}$ projects to a walk in $G^*$ by dropping subscripts; in particular, any walk in $\overline{G}$ from $v_0$ to $v_h$ projects to a closed walk in $G^*$ with homology class $h$ that starts and ends at vertex $v$.
Conversely, the \emph{shortest} closed walk in $G^*$ in any homology class $h$ is the projection of the \emph{shortest} path from $v_0$ to $v_h$, for some vertex $v$.

Any cycle in any non-trivial homology class crosses some path $\alpha_i$ an odd number of times, and therefore at least once.  To find all such cycles for each index~$i$, we slice~$\overline{G}$ along the lifted path $\alpha_{i,0}$ to obtain a new boundary cycle, and then compute the shortest path from each vertex $v_0$ on this cycle to every other vertex $v_h$ in $2^{O(g)}n\log n$ time, using an algorithm of Chambers \etal~\cite{cce-msspe-13}.  Altogether, we find the shortest closed walk in every $\mathbb{Z}_2$-homology class in $2^{O(g)}n\log n$ time.  The dual minimum cut~$X^*$ can then be built from these $\mathbb{Z}_2$-minimal cycles  in $2^{O(g)}$ additional time via dynamic programming.

\subsection{Global Minimum Cuts}

Our final result generalizes the recent $O(n\log \log n)$-time algorithm for planar graphs by Łącki and Sankowski \cite{ls-mcsc-11}, which in turn relies on the $O(n\log\log n)$-time algorithm for planar minimum $(s,t)$-cuts of Italiano \etal\ \cite{insw-iamcmf-11}.

The global minimum cut~$X$ in a surface graph $G$ is dual to the minimum-cost nonempty separating subgraph of the dual graph $G^*$.  In particular, if $G$ is planar,~$X$ is dual to the shortest nonempty cycle in $G^*$.
There are two cases to consider: either~$X^*$ is a simple contractible cycle, or it isn’t.  We describe two algorithms, one of which is guaranteed to return the minimum-cost separating subgraph.

To handle the contractible cycle case, we first slice the surface $\Sigma$ to make it planar, first along the shortest non-separating cycle $\alpha$ in $G^*$, which we compute in $g^{O(g)}n\log\log n$ time using a variant of our crossing sequences algorithm, and then along a greedy system of arcs $A$ connecting the resulting boundary cycles.
Call the resulting planar graph $D$; each edge of $\alpha\cup A$ appears as two edges on the boundary of $D$.  
Let $e^+$ and $e^-$ be edges on the boundary of $D$ corresponding to some edge $e$ of $\alpha$.  Using the planar algorithm of Łącki and Sankowski~\cite{ls-mcsc-11}, we find the shortest cycle $\gamma^+$~in $D\setminus e^+$ and the shortest cycle $\gamma^-$ in $D\setminus e^-$.  The shorter of these two cycles projects to a closed walk $\gamma$ in the original dual graph $G^*$.  Results of Cabello \cite{c-fscss-10} imply that if $\gamma$ is a simple cycle, it is the shortest contractible simple cycle in $G^*$; otherwise,~$X^*$ is not a simple cycle.

Our second algorithm begins by enumerating all $2^{O(g)}$ $\mathbb{Z}_2$-minimal even subgraphs in $G^*$ in $g^{O(g)}n\log\log n$ time, using our crossing sequence algorithm.
Our algorithm marks the faces on either size of an arbitrary edge of each $\mathbb{Z}_2$-minimal even subgraphs in $G^*$.  If $X^*$ is not a simple contractible cycle, then some pair of marked faces must be separated by $X^*$.  In other words, in $g^{O(g)}n\log\log n$ time, we identify a set $T$ of $2^{O(g)}$ vertices of $G$, at least two of which are separated by the global minimum cut.  Thus, if we fix an arbitrary source vertex $s$ and compute the minimum $(s,t)$-cut for each vertex $t\in T$ in $g^{O(g)}n\log\log n$ time, the smallest such cut is the global minimum cut $X$.


% ----------------------------------------------------------------------

\OpenProb

Extending these algorithms to \emph{directed} surface graphs remains an interesting open problem; currently the only effective approach known is to compute a maximum $(s,t)$-flow and apply the maxflow-mincut theorem.  The recent algorithm of Borradaile and Klein~\cite{bk-amfdp-09} computes maximum flows in directed \emph{planar} graphs in $O(n\log n)$ time. For higher-genus graphs, Chambers \etal~\cite{cen-hfcc-12} describe maximum-flow algorithms that run in $g^{O(g)}n^{3/2}$ time for arbitrary capacities and in $O(g^8 n\log^2 n \log^2 C)$ for integer capacities that sum to $C$.

Another open problem is reducing the dependencies on the genus from exponential to polynomial.  Even though there are near-quadratic algorithms to compute minimum cuts, the only known approach to achieving near-linear time for bounded-genus graphs with weighted edges is to solve an NP-hard problem!

Finally, it is natural to ask whether minimum cuts can be computed quickly in other minor-closed families of graphs, for which embeddings on to bounded-genus surfaces may not exist.
Such results already exist for one-crossing-minor-free families~\cite{ce-fog-13} and in particular, graphs of bounded treewidth~\cite{hknr-cmfnc-98}.


% ----------------------------------------------------------------------

\CrossRef

Max cut\\
Sparsest cut\\
Separators in graphs\\
Shortest paths in planar graphs with negative weight edges


% ----------------------------------------------------------------------

\nocite{e-cocb-12}

\bibliographystyle{spbasic}
%\bibliography{jeffe,topology,data-structures,optimization}
\bibliography{surfcutEoA}


\end{document}
