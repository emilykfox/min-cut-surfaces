\documentclass[natbib]{svcyclop}
\usepackage{graphicx} 
\usepackage{amsfonts,amsmath}
\usepackage{hyperref}
\usepackage{url}
\urlstyle{tt}

\usepackage[utf8x]{inputenc}	% Understand Unicode in source file
\usepackage[T1]{fontenc}	% OT1 does not support ogonek (\k) in Łącki = \L\k{a}cki
\usepackage{microtype}		% Oh, yes, thank GOD.
\usepackage{stmaryrd,marvosym}

\usepackage[dvipsnames,usenames]{color}
\def\NOTE#1{\textcolor{Red}{\textbf{\textsf{••• #1 •••}}}}

%-----------------------------------------------------------------------
%  Local definitions
%-----------------------------------------------------------------------
\def\arcto{\mathord\shortrightarrow}
\def\arc#1#2{#1\arcto#2}
\def\cra#1#2{#1\mathord\shortleftarrow#2}
\def\fence#1#2{#1\mathord\shortuparrow#2}
\def\ecnef#1#2{#1\mathord\shortdownarrow#2}
\def\head{\mathit{head}}
\def\tail{\mathit{tail}}
\def\lsh{\mathit{left}}
\def\rsh{\mathit{right}}
\def\rev{\mathit{rev}}
\def\Z{\mathbb{Z}}
\def\Real{\mathbb{R}}
\def\Q{\mathbb{Q}}
\def\reverse#1{\smash{\overline{#1}}}
\def\snip{\mathbin{\raisebox{0.15ex}{\rotatebox[origin=c]{60}{\Rightscissors}\!}}}
\def\subsnip{\mathbin{\raisebox{0.15ex}{\rotatebox[origin=c]{60}{\footnotesize\Rightscissors}\!}}}
\def\Gsnip{\mathord{G_{\subsnip}}}
\def\Sigmasnip{\mathord{\Sigma_{\subsnip}}}
\def\gammasnip{\mathord{\gamma_{\subsnip}}}

% needs marvosym package:
\def\snip{\mathbin{\raisebox{0.15ex}{\rotatebox[origin=c]{60}{\Rightscissors}\!}}}
\let\unlhd\trianglelefteq	% fix mathdesign font bug

\let\cycle\gamma
\let\path p
\let\primalarc\alpha
\def\dualarc{p}

\def\Sigmabar{\overline{\smash{\Sigma}\vphantom{t}}}
\def\SIGMABAR{\boldsymbol{\overline{\smash{\Sigma}\vphantom{x}}}}
\def\Gbar{\overline{\smash{G}\vphantom{t}}}
\def\Vbar{\overline{\smash{V}\vphantom{t}}}
\def\Ebar{\overline{\smash{E}\vphantom{t}}}
\def\bbar{\overline{\smash{b}\vphantom{t}}}
\def\nbar{\overline{n}}
\def\gbar{\overline{g}}
\def\wbar{\overline{w}}
\def\sigmabar{\overline{\sigma}}
\def\cyclebar{\overline{\cycle}}
\def\chibar{\overline{\chi}}

\def\fakeparagraph#1{\par\medskip\noindent\textbf{#1}}


%\newtheorem{theorem}{Theorem}[section]
%\newtheorem{corollary}[theorem]{Corollary}
%\newtheorem{lemma}[theorem]{Lemma}

% ----------------------------------------------------------------------
\begin{document}

\title{Cuts in surface-embedded graphs}

% - - - - - - - - - - - - - - - - - - - - - - - - - - - - - - - - - - - - - - - - 

\author{
	Erin W. Chambers\inst{1}
\and
	Jeff Erickson\inst{2}
\and
	Kyle Fox\inst{3}
\and
	Amir Nayyeri\inst{4}}

% - - - - - - - - - - - - - - - - - - - - - - - - - - - - - - - - - - - - - - - - 

\institute{
	Department of Computer Science and Mathematics,
	Saint Louis University, St. Louis, MO, USA
\and
	Department of Computer Science,
	University of Illinois, Urbana, IL, USA
\and
	Institute for Computational and Experimental Research in Mathematics,
	Brown University, Providence, RI, USA
\and
	Department of Electrical Engineering and Computer Science,
	Oregon State University, Corvallis, OR, USA}

% - - - - - - - - - - - - - - - - - - - - - - - - - - - - - - - - - - - - - - - - 

\sumoriwork{2009; Chambers, Erickson, Nayyeri\\
2011; Erickson, Nayyeri\\
2012; Erickson, Fox, Nayyeri}

% - - - - - - - - - - - - - - - - - - - - - - - - - - - - - - - - - - - - - - - - 

\keywords{Topological graph theory;
Graph embedding;
Minimum cuts; 
Homology;
Covering spaces;
Fixed-parameter tractability}

\maketitle

% ----------------------------------------------------------------------

\ProbDef

\subsection{Background}

%\begin{itemize}
%\item Surfaces, genus, embedding, duality
%\end{itemize}

A surface is a compact Hausdorff space in which each point has a neighborhood homeomorphic to the plane or closed half plane.
The genus of a surface is the maximum number of disjoint simple cycles whose complement with respect to the surface is connected.
An embedding of a graph on a surface maps vertices to distinct points on the surface and edges to interior disjoint curves.  
A face of an embedding is a maximal connected subset of the surface that is disjoint from the image of the graph.
An embedding is cellular if every face is homeomorphic to the plane.


\subsection{Problem statement}

\NOTE{Kyle: I think the first and third papers only address orientable surfaces, and all our magic numbers assume orientable.}
\subsubsection{Problem 1 (Minimum (s,t)-Cut)}
{\itshape
\textsc{Input}: An undirected graph $G = (V, E)$ embedded on a surface of genus $g$, a non-negative capacity function $c\colon E \to \mathbb{R}$, and two vertices $s$ and $t$. \textsc{Output}:~A minimum-capacity $(s,t)$-cut in $G$.}


\subsubsection{Problem 2 (Global Minimum Cut)}
{\itshape
\textsc{Input}: An undirected graph $G = (V, E)$  embedded on a surface of genus $g$ and a non-negative capacity function $c\colon E \to \mathbb{R}$.  \textsc{Output}: A minimum-capacity cut in $G$.}



% ----------------------------------------------------------------------

\KeyRes

\subsection{Topological Background}

%\begin{itemize}
%\item Duality
%\item Planar minimum cut is dual to shortest generating cycle in an annulus \cite{is-mfpn-79, r-mstcp-83}
%\item Homotopy: continuous deformation
%\item Homology: even subgraphs, boundary subgraphs
%\item Tree-cotree decompositions, greedy systems of loops and arcs
%\item Covering spaces
%\end{itemize}

The dual of a cellularly embedded graph $G$ is a graph $G^*$.  The vertices of $G^*$ correspond to the faces of $G$, and two vertices (in $G^*$) are joined by a (dual) edge if and only if they correspond to adjacent faces of $G$.
We denote the dual of a set of edges~$X$ as~$X^*$.
Duality maps cuts to certain sets of cycles and vice versa.
In particular, a planar minimum $(s,t)$-cut is dual to shortest generating cycle in an annulus \cite{is-mfpn-79, r-mstcp-83}.

Two cycles are homotopic if one can be continuously deformed to the other.  
Two even subgraphs are $\mathbb{Z}_2$-homologous if their symmetric difference is the boundary of a subset of the surface.
An even subgraph (or cycle) is $\Z_2$-minimal if its edges have minimum total weight within its homology class.
Any even subgraph can be decomposed into cycles (closed walks).
Each cycle of a $\Z_2$-minimal even subgraph $X$ is itself $\Z_2$-minimal; however, these individual cycles could lie in arbitrary homology classes.

A tree-cotree decomposition is a decomposition of the edges of a cellularly embedded graph into three subsets $(T, C, X)$ where $T$ is a spanning tree of $G$, $C^*$ is a spanning tree of $G^*$ and $X$ is a set of $2g$ edges.  
Adding each member of $X$ to $T$ gives a unique cycle.  
If $T$ is a shortest path tree, the set of $2g$ cycles that is obtained by adding members of $X$ to $T$ is called a greedy system of loops.
There is a similar concept called a greedy system of arcs for surfaces with boundary.
\NOTE{Kyle: And now do we define boundary and arcs, or just let it be?}

A covering of a surface $\Sigma$ is a space $\overline{\Sigma}$ together with a continuous surjective map, $\sigma: \overline{\Sigma} \rightarrow \Sigma$. For every point $p \in \Sigma$ there is a neighborhood $U$ of $p$ such that $\sigma^{-1}(U)$ is a union of disjoint discs in $\overline{\Sigma}$.

Let $X$ denote the set of edges that cross the minimum $(s,t)$-cut in $G$.  Then $X^*$ induces the minimum-cost subgraph of $G^*$ that is $\Z_2$-homologous with $\partial s^*$.


\subsection{Crossing Sequences}

%\begin{itemize}
%\item
%Our first algorithm \cite{surfcut} reduces computing a minimum $(s,t)$-cut in a graph embedded on a genus-$g$ surface to $g^{O(g)}$ instances of the planar minimum-cut problem.
%\item
%Any cycle decomposition of any $\Z_2$-minimal subgraph crosses any shortest path, and therefore any arc in the greedy system of arcs, at most $O(g)$ times.
%\item
%Cutting the surface along the system of arcs and contracting the resulting boundary arcs to points yields a polygon with $O(g)$ vertices called an \emph{abstract polygonal schema}.
%\item
%The crossing pattern of any even subgraph can be encoded as a weighted triangulation of this polygon.  We only need to consider triangulation with edge weights between $0$ and $O(g)$; there are $g^{O(g)}$ such weighted triangulations.
%\item
%Conversely, every weighted triangulation corresponds to a homotopy class of a collection of cycles.  This collection of cycles comprises the dual of an $(s,t)$-cut if and only if the total weight incident to every vertex of the triangulation is odd.  More generally, the parities of the weight incident to the vertices define a vector that encodes the homology class of the cycles.
%\item
%From each weighted triangulation, we can find the corresponding minimum-cost even subgraph using an algorithm of Kutz \cite{k-csnco-06}.  The triangulation itself encodes a decomposition of the target subgraph into edge-disjoint cycles.  Each of these cycles is the shortest generating cycle in an annulus obtained by gluing $O(g^2)$ copies of the disk $\Sigma\setminus P$, and can therefore be computed in $O(g^2n\log\log n)$ time using the planar minimum-cut algorithm of Italiano \etal~\cite{insw-iamcmf-11}.
%\item
%The overall running time of the algorithm is $g^{O(g)}n\log\log n$.
%\item
%As a side result, we prove that finding the minimum-cost even subgraph in any $\Z_2$-homology class is NP-hard, by a reduction from \textsc{MaxCut}.  Different reductions imply that it is NP-hard to find the minimum-cost \emph{closed walk} \cite{splitting} or \emph{simple cycle}
%\cite{ccl-fctpe-11} in a given $\Z_2$-homology class.
%\end{itemize}

Our first algorithm \cite{cen-mcshc-09} reduces computing a minimum $(s,t)$-cut in a graph embedded on a genus-$g$ surface to $g^{O(g)}$ instances of the planar minimum-cut problem.
Recall that the minimum $(s,t)$-cut is dual to the minimum even subgraph in a certain homology class.
Any cycle decomposition of any $\Z_2$-minimal subgraph crosses any shortest path, and therefore any arc in the greedy system of arcs, at most $O(g)$ times.
We enumerate over all possible collections of crossing sequences between members of such a cycle decomposition and the arcs in the greedy system of arcs.
The crossing bound above is exploited to restrain the number of possibilities to at most~$g^{O(g)}$.

From each possible collection of crossing sequences, we can find the corresponding minimum-cost even subgraph using an algorithm of Kutz \cite{k-csnco-06}.
Each cycle in any~$\Z_2$-minimal subgraph is the shortest generating cycle in an annulus obtained by gluing $O(g^2)$ copies of the disk $\Sigma\setminus P$, and can therefore be computed in $O(g^2n\log\log n)$ time using the planar minimum-cut algorithm of Italiano \etal~\cite{insw-iamcmf-11}.
The overall running time of the algorithm is $g^{O(g)}n\log\log n$.

Surprisingly, finding the minimum-cost even subgraph in an arbitrary $\Z_2$-homology class is NP-hard, by a reduction from \textsc{MaxCut}.  Different reductions imply that it is NP-hard to find the minimum-cost \emph{closed walk} \cite{ccelw-scsih-08} or \emph{simple cycle}~\cite{c-fscss-10} in a given $\Z_2$-homology class.

\subsection{$\Z_2$-Homology Cover}

%\begin{itemize}
%\item
%Our second algorithm \cite{homcover} explicitly searches for all $\Z_2$-minimal cycles in $G^*$, and then assembles them into the dual of the minimum $(s,t)$-cut.
%\item
%As in our first algorithm, we first cut the surface into a disk using a greedy system of arcs $A$.  The homology class of a cycle $\gamma$ is determined by the parity of the number of crossings of $\gamma$ with each arc in $A$.
%\item
%The \emph{$\Z_2$-homology cover} $\overline\Sigma$ of the combinatorial surface $\Sigma$ is defined by assembling several copies of $\Sigma$ as follows.  First, cut the surface $\Sigma$ along all $2g+1$ arcs in $A$ to obtain a topological disk~$D$; each arc $\alpha_i\in A$ appears as two paths $\alpha_i^+$ and $\alpha_i^-$ on the boundary of $D$.  For each index $i$ and homology class $h\in\Z_2^{2g+1}$, let $\alpha^+_{i,h}$ and $\alpha^-_{i,h}$ respectively denote the copies of $\alpha_i^+$ and $\alpha_i^-$ in $D_h$.  Let $h\oplus h’$ denote the bit-wise exclusive-or of two bit vectors $h, h’\in\Z_2^{2g+1}$, and let $h \land i = h\oplus 2^i$ denote the bit vector obtained from $h$ by flipping its $i$th bit.  Finally, $\overline\Sigma$ is constructed by gluing together these $2^{2g+1}$ disks by identifying the paths $\alpha^+_{i,h}$ and $\alpha^-_{i,h \land i}$ for each homology class $h$ and index $i$.  
%
%\item
%This combinatorial surface has complexity $2^{O(g)}n$ and genus $2^{O(g)}$.  In particular, 
%every vertex $v$ and edge $e$ induces $2^{2g+1}$ vertices $v_h$ and edges $e_h$ in $\overline{G}$, one for each homology class $h$.  Each edge $e_h$ of $\overline{G}$ inherits the cost of the corresponding edge $e$ in $G$.  
%
%\item
%Any walk in $\overline{G}$ projects to a walk in $G$ (by dropping subscripts); conversely, any walk in $G$ lifts to $2^{2g+1}$ different walks in $\overline{G}$, each uniquely determined by a lift of one of its endpoints.  In particular, any walk in $\overline{G}$ from $v_h$ to $v_{h’}$ projects to a closed walk in $G$, starting and ending at $v$, with homology class $h\oplus h’$.  Conversely, the \emph{shortest} closed walk in any homology class $h$ is the projection of the \emph{shortest} path from $v_0$ to $v_h$, for some vertex $v$.
%
%\item
%Apply multiple-source shortest paths \cite{multishort} to compute this shortest path in $2^{O(g)}n\log n$ time. \NOTE{Need more detail here.}
%\end{itemize}


Our second algorithm \cite{en-mcsnc-11} explicitly searches for all $\Z_2$-minimal cycles in $G^*$, and then assembles them into the dual of the minimum $(s,t)$-cut.
As in our first algorithm, we first cut the surface into a disk using a greedy system of arcs $A$.  The homology class of a cycle $\gamma$ is determined by the parity of the number of crossings of $\gamma$ with each arc in $A$.

Our algorithm reduces the problem of computing minimum cycles in certain homology classes to computing shortest paths in a larger covering space obtained from cutting and gluing multiple copies of $\Sigma$.
The \emph{$\Z_2$-homology cover} $\overline\Sigma$ of the combinatorial surface $\Sigma$ is defined by assembling several copies of $\Sigma$ as follows.  First, cut the surface $\Sigma$ along all $2g+1$ arcs in $A$ to obtain a topological disk~$D$; each arc $\alpha_i\in A$ appears as two paths $\alpha_i^+$ and $\alpha_i^-$ on the boundary of $D$.  For each index $i$ and homology class $h\in\Z_2^{2g+1}$, let $\alpha^+_{i,h}$ and $\alpha^-_{i,h}$ respectively denote the copies of $\alpha_i^+$ and $\alpha_i^-$ in $D_h$.  Let $h\oplus h’$ denote the bit-wise exclusive-or of two bit vectors $h, h’\in\Z_2^{2g+1}$, and let $h \land i = h\oplus 2^i$ denote the bit vector obtained from $h$ by flipping its $i$th bit.  Finally, $\overline\Sigma$ is constructed by gluing together these $2^{2g+1}$ disks by identifying the paths $\alpha^+_{i,h}$ and $\alpha^-_{i,h \land i}$ for each homology class $h$ and index $i$.  

This combinatorial surface has complexity $2^{O(g)}n$ and genus $2^{O(g)}$.  In particular, 
every vertex $v$ and edge $e$ induces $2^{2g+1}$ vertices $v_h$ and edges $e_h$ in $\overline{G}$, one for each homology class $h$.  Each edge $e_h$ of $\overline{G}$ inherits the cost of the corresponding edge $e$ in $G$.  
Any walk in $\overline{G}$ projects to a walk in $G$ (by dropping subscripts); conversely, any walk in $G$ lifts to $2^{2g+1}$ different walks in $\overline{G}$, each uniquely determined by a lift of one of its endpoints; all have the same length.  In particular, any walk in $\overline{G}$ from $v_h$ to $v_{h’}$ projects to a closed walk in $G$, starting and ending at $v$, with homology class $h\oplus h’$.  Conversely, the \emph{shortest} closed walk in any homology class $h$ is the projection of the \emph{shortest} path from $v_0$ to $v_h$, for some vertex $v$.
Therefore, knowing $v$ would allow us to compute the shortest closed walk.

The shortest cycle in each non-zero homology class crosses at least one arc of $A$ an odd number of times; in particular, at least once.
Let $a$ denote such an arc and $\overline{a}$ denote any lift of it in $\overline{G}$.  We cut $\overline{G}$ along $\overline{a}$ to obtain a face whose boundary is composed of two copies of $\overline{a}$.
Finally, we apply multiple-source shortest paths \cite{cce-msspe-13} to compute all shortest paths with one endpoint on $\overline{a}$ in $2^{O(g)}n\log n$ time.

\subsection{Global Minimum Cuts}

Our final result generalizes the recent $O(n\log \log n)$-time algorithm for planar graphs by Łącki and Sankowski \cite{ls-mcsc-11}, which in turn relies on the $O(n\log\log n)$-time algorithm for planar minimum $(s,t)$-cuts of Italiano \etal\ \cite{insw-iamcmf-11}.

The global minimum cut~$X$ in a surface graph $G$ is dual to the minimum-cost nonempty separating subgraph of the dual graph $G^*$.  In particular, if $G$ is planar,~$X$ is dual to the shortest nonempty cycle in $G^*$.
There are two cases to consider: either~$X^*$ is a simple contractible cycle, or it isn’t.  We describe two algorithms, one of which is guaranteed to return the minimum-cost separating subgraph.


To handle the contractible cycle case, we first slice the surface $\Sigma$ to make it planar.  Specifically, we first slice the surface along the shortest non-separating cycle $\alpha$ in $G^*$ so it contains two boundary components; we can find this cycle in $g^{O(g)}n\log\log n$ time using a variant of our crossing sequences algorithm.
We then further slice the surface along a greedy system of arcs connecting its two boundary components in $O(gn)$ time.
Call the resulting planar graph $D$; each edge of the cycle $\alpha$ and the greedy system of arcs appears as two edges on the outer face of $D$.  Results of Cabello \cite{c-fscss-10} imply that the shortest simple contractible cycle in  $G^*$ is also a simple cycle in $H$.  Let $e$ be an arbitrary edge of the cycle $\alpha$; this edge appears as two edges $e_1$ and $e_2$ on the outer face of $D$.  Using the planar algorithm of Łącki and Sankowski \cite{ls-mcsc-11}, we find the shortest cycle $\gamma’_1$ in  $D\setminus e_1$ and the shortest cycle $\gamma’_2$ in $D\setminus e_2$.  Let $\gamma’$ be the shorter of these two cycles; this cycle projects to a closed walk $\gamma$ in the original dual graph $G^*$.  If $\gamma$ is a simple cycle, then $\gamma$ is the shortest contractible simple cycle in $G^*$; otherwise,~$X$ is not a simple cycle.


Now suppose the minimum-cost separating subgraph $X^*$ is not a simple contractible cycle.  Our second algorithm begins by enumerating all $2^{O(g)}$ $\Z_2$-minimal even subgraphs in $G^*$ in $g^{O(g)}n\log\log n$ time using our crossing sequence algorithm.  Now let $A$ and $B$ denote the components of the $\Sigma$ to either side of $X^*$.  Because $X^*$ is not a contractible cycle, the surfaces $A$ and $B$ both have nontrivial topology, and therefore each contain at least one nontrivial $\Z_2$-minimal even subgraph.  All such subgraphs are also $\Z_2$-minimal in~$G^*$.  For any edge $e$ in any $\Z_2$-minimal even subgraph in $A$, at least one face of $\Sigma$ incident to $e$ is also a face of $A$.  Thus, if we mark the faces on either size of an arbitrary edge of each $\Z_2$-minimal even subgraph in $G^*$, the resulting set of $2^{O(g)}$ marked faces include at least two that are separated by~$X^*$.  In other words, in $g^{O(g)}n\log\log n$ time, we can identify a set $T$ of $2^{O(g)}$ vertices of $G$, at least two of which are separated by the global minimum cut.  Thus, if we fix an arbitrary source vertex $s$ and compute the minimum $(s,t)$-cut for each vertex $t\in T$ in $2^{O(g)}\cdot g^{O(g)}n\log\log n = g^{O(g)}n\log\log n$ time, the smallest such cut is the global minimum cut.



% ----------------------------------------------------------------------

\OpenProb

Extending these minimum-cut algorithms to \emph{directed} surface graphs remains an interesting open problem.  Indeed, the most efficient algorithms known to compute minimum $(s,t)$-cuts in directed surface graphs first compute a maximum $(s,t)$-flow and apply the maxflow-mincut theorem.  In \emph{planar} directed graphs, maximum $(s,t)$-flows can be computed in $O(n\log n)$ time using the recent algorithm of Borradaile and Klein~\cite{bk-amfdp-09}; minimum $(s,t)$-cuts can also be computed directly in $O(n\log n)$ time using the covering-space algorithm of \cite{wobble}.
\NOTE{Kyle: What is references wobble supposed to be?}
For higher-genus graphs, Chambers \etal~\cite{cen-hfcc-12} describe a  maximum-flow algorithms that run in $g^{O(g)}n^{3/2}$ time for arbitrary capacities and in $O(g^8 n\log^2 n \log^2 C)$ for integer capacities that sum to $C$.

Another open problem is reducing the dependence on the genus from exponential to polynomial.  Even though there are near-quadratic algorithms to compute minimum cuts, the only known approach to achieving near-linear time for bounded-genus graphs with weighted edges is to solve an exponential number of instances of an NP-hard problem!

Finally, it is natural to ask whether minimum cuts can be computed quickly in other minor-closed families of graphs, for which embeddings on to bounded-genus surfaces may not exist.
\NOTE{Kyle: Citations for tree-width and one-crossing-minor free flow go here, room permitting.}


% ----------------------------------------------------------------------

\CrossRef

\NOTE{Check table of contents of volume 2}

% ----------------------------------------------------------------------

\nocite{bk-amfdp-09, hkrs-fspap-97, k-msspp-05, r-mstcp-83, insw-iamcmf-11, multishort, parshort, splitting, gohog, optcycles, surflow, surfcut, homcover, global, e-dgteg-03, c-scgsp-10, p-deeoc-13}

\bibliographystyle{spbasic}
%\bibliography{jeffe,topology,data-structures,optimization}
\bibliography{../topology,../data-structures,../optimization}


\end{document}
